\documentclass[12pt]{article}
\usepackage{wkrpt}
\begin{document}


\title{Analysis of Web Application Functional Testing Options}
{
	Telus Health Inc.\\
	Mississauga, ON L4W 4T9
}
{
	\textbf{Prepared by}\\[2ex]
	
	Daiwei Fan\\
	Student ID: 20458752\\
	User ID: d3fan\\
	2A Software Engineering\\
	April 11, 2014
}


\letter{Analysis of Web Application Functional Testing Options}{first}{2A}{Telus Health Inc.}
{
	\noindent
	Daiwei Fan\\
	\#48, 461 Beechwood Place\\
	Waterloo, ON\\
	N2T 2N8
}
{
	Telus Health provides health and related transaction record for clients of medical insurance companies. I was situated in the new VIP room team, which is responsible for rebuilding a web service for clients to look up their drug claim transaction history. During my time in Telus, I was assigned to study the pre-existing interface and to extract data from existing log files.
}
{
	The following report performs an analysis on the different techniques for testing features implemented in VIP room. This problem is one that I encountered during my work term. My team had to decide which method was best suited to the task given all the requirements and our abilities.
}
{
	I would like to thank my manager Kevin Ho and Quality Analyst Anshuman Ghandi, both of whom gave inputs on how functionality testing is done in other projects produced by Telus Health.
}
{
	Daiwei Fan\\
	Student ID: 20458752\\[2ex]
	Encl.
}


\tocsection{Executive Summary}
EXECUTIVE SUMMARY

\begin{figure}[ht!]
\centering
\includegraphics[width=90mm,height=5cm,keepaspectratio]{img/bbb.jpg}
\caption{A simple caption}
\label{overflow}
\end{figure}

\begin{figure}[ht!]
\centering
\includegraphics[width=90mm,height=5cm,keepaspectratio]{img/aaa.jpg}
\caption{A simple caption}
\label{overflow}
\end{figure}


\newpage

\toc
% \lof
% \lot


\pagenumbering{arabic}
\section{Introduction}

Telus Health is a “TELUS Health is a leader in telehomecare, electronic medical and health records, consumer health, benefits management and pharmacy management.”\cite{telusComp}. With the recent acquisition of Med Access Inc., Telus is “confirmed as the largest EMR (Electronic Medical Record) provider in Canada”\cite{telusComp}. This title has encouraged Telus to provide service with higher quality. In January 2014, I was assigned to the VIP room to implement a web service that mainly contains a drug claim searching and benefit management feature. This particular feature allows different levels of users such as patient, insurer and pharmacist to access the patient's record of drug claims and payments. After a certain amount of designing and coding, it came to the point where some basic testing for the code was necessary. I believe that even elementary testings should be carefully designed for future documentation and reproduction. \\

Functionality testing has always been one of the most important tasks in the quality assurance (abbreviated as QA) process. All companies are working to discover a testing method which is able to cover all sources of error, requires minimal human effort and consumes least amount of time. It is especially important for Telus Health, a company which has adapted the Agile software development system, to have a highly efficient QA process in order to accommodate the fast pace of the system.\\

Since the advent of computers, people are persistently finding new and innovative 
ways of automating as many tasks and processes as possible. Automation is a desirable 
goal for many companies because it is a very effective way of increasing efficiency. 
Once a process is automated, a company need not allocate a lot, if any, resources to 
executing that process since it is now performed without any manual intervention. It then 
follows that Goodbank, a bank whose name has been changed because it is not needed for 
the report and for privacy reasons, will want to automate a significant number of tasks. 
This is especially true in the case of quality assurance (QA) because of the importance it 
holds for financial companies, like banks, and the amount of resources that banks 
typically require for QA assignments. It is important to note that “(Manual) Testing 
accounts for 25% to 50% of the total budget in many software projects [1]” which is no 
exception in the case of Goodbank. One group of processes that Deloitte has been 
brought in to automate are the QA tasks that involve testing the messages that are sent 
between the many systems utilized by Goodbank. These messages are used to relay 
information and commands between the bank’s services. All of these messages are in 
Extensible Markup Language (XML) format but differ considerably in the layout of the 
elements. All of the respective systems know how to interpret and respond to messages 
that fit each system’s specified format; all other messages are ignored and discarded. 
Currently, the QA department of Goodbank tests the messaging services by manually 
creating XML request messages that satisfy certain test requirements, comparing the 
response messages they receive, also in XML format, and determining if there are any 
discrepancies between the response message and the expected response message. Manual 
testing, such as the process just described, is often considered “difficult, tedious, time 
consuming, and inadequate [1]”. This is the task that the Deloitte team has been brought 
in to automate.



\newpage

\section{Problem Specification}
background
\subsection{Options}

\section{Comparison and Evaluation}
ccc


\section{Conclusions}
CONCLUSIONS


\section{Recommendations}
RECOMMENDATIONS


\newpage


\addcontentsline{toc}{section}{\refname}
\bibliography{wkrpt}
\begin{thebibliography}{1}

  \bibitem{telusComp} Telus Health Inc., "Telus Health: Home," Telus Health Inc., {\em http://www.telushealth.com/} (current April 11 2014).

  \bibitem{impj}  The Japan Reader {\em Imperial Japan 1800-1945} 1973:
  Random House, N.Y.

  \bibitem{norman} E. H. Norman {\em Japan's emergence as a modern
  state} 1940: International Secretariat, Institute of Pacific
  Relations.

  \bibitem{fo} Bob Tadashi Wakabayashi {\em Anti-Foreignism and Western
  Learning in Early-Modern Japan} 1986: Harvard University Press.

\end{thebibliography}
\newpage


\tocsection{Acknowledgements}
ACKNOWLEDGEMENTS
\newpage


% \appendix{APPENDIX INDEX}{APPENDIX NAME}
% APPENDICES
% \newpage


\end{document}

